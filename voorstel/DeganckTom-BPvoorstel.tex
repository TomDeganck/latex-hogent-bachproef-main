%==============================================================================
% Sjabloon onderzoeksvoorstel bachproef
%==============================================================================
% Gebaseerd op document class `hogent-article'
% zie <https://github.com/HoGentTIN/latex-hogent-article>

% Voor een voorstel in het Engels: voeg de documentclass-optie [english] toe.
% Let op: kan enkel na toestemming van de bachelorproefcoördinator!
\documentclass{hogent-article}

% Invoegen bibliografiebestand
\addbibresource{voorstel.bib}

% Informatie over de opleiding, het vak en soort opdracht
\studyprogramme{Professionele bachelor toegepaste informatica}
\course{Bachelorproef}
\assignmenttype{Onderzoeksvoorstel}
% Voor een voorstel in het Engels, haal de volgende 3 regels uit commentaar
% \studyprogramme{Bachelor of applied information technology}
% \course{Bachelor thesis}
% \assignmenttype{Research proposal}

\academicyear{2024-2025} % TODO: pas het academiejaar aan

% TODO: Werktitel
\title{Real-Time Vertaling van Vlaamse Gebarentaal naar Tekst met Computer Vision}

% TODO: Studentnaam en emailadres invullen
\author{Tom Deganck}
\email{tom.deganck@student.hogent.be}

% TODO: Medestudent
% Gaat het om een bachelorproef in samenwerking met een student in een andere
% opleiding? Geef dan de naam en emailadres hier
% \author{Yasmine Alaoui (naam opleiding)}
% \email{yasmine.alaoui@student.hogent.be}

% TODO: Geef de co-promotor op
\supervisor[Co-promotor]{}

% Binnen welke specialisatierichting uit 3TI situeert dit onderzoek zich?
% Kies uit deze lijst:
%
% - Mobile \& Enterprise development
% - AI \& Data Engineering
% - Functional \& Business Analysis
% - System \& Network Administrator
% - Mainframe Expert
% - Als het onderzoek niet past binnen een van deze domeinen specifieer je deze
%   zelf
%
\specialisation{AI \& Data Engineering}
\keywords{Kunstmatige Intelligentie, Computer vision,Gebarentaal, Vertalen}

\begin{document}

\begin{abstract}
De nood aan voor toegankelijkheid voor doven en slechthorenden is de laatste jaren sterk toegenomen. 
De Vlaamse Gebarentaal (VGT) is de taal die gebruikt wordt door de Vlaamse Dovengemeenschap. 
Deze taal is echter niet gekend door de meeste mensen. 
Dit maakt het moeilijk voor doven en slechthorenden om te communiceren met de horende wereld. 
Dit onderzoek zal zich focussen op het real-time vertalen van VGT naar tekst. 
Hiervoor zal gebruik gemaakt worden van computer vision en deep learning.
De bedoeling is om een neuraal netwerk te ontwikkelen dat in staat is om VGT te herkennen en te vertalen naar tekst. 
Dit zal gebeuren door middel van een video die de gebaren van de gebruiker herkent en deze vertaalt naar tekst. 
Hierbij zal gebruik gemaakt worden van een deep learning model dat getraind wordt op een dataset van VGT gebaren.
Het doel van dit onderzoek is om een systeem te ontwikkelen dat in staat is om VGT te vertalen naar tekst in real-time. 
Dit zal het voor doven en slechthorenden makkelijker maken om te communiceren met de horende wereld.
\end{abstract}

\tableofcontents

% De hoofdtekst van het voorstel zit in een apart bestand, zodat het makkelijk
% kan opgenomen worden in de bijlagen van de bachelorproef zelf.
%---------- Inleiding ---------------------------------------------------------

% TODO: Is dit voorstel gebaseerd op een paper van Research Methods die je
% vorig jaar hebt ingediend? Heb je daarbij eventueel samengewerkt met een
% andere student?
% Zo ja, haal dan de tekst hieronder uit commentaar en pas aan.

%\paragraph{Opmerking}

% Dit voorstel is gebaseerd op het onderzoeksvoorstel dat werd geschreven in het
% kader van het vak Research Methods dat ik (vorig/dit) academiejaar heb
% uitgewerkt (met medesturent VOORNAAM NAAM als mede-auteur).
% 
\section{Inleiding}%
\label{sec:inleiding}

\subsection{Vlaams Gebarentaal}
\label{sec:VGT}

In de hedendaagse samenleving is er een groeiende behoefte aan technologieën die de communicatie tussen doven en horenden vergemakkelijken. Een van de meest veelbelovende ontwikkelingen op dit gebied is de real-time vertaling van Vlaamse Gebarentaal (VGT) naar tekst, waarbij gebruik wordt gemaakt van kunstmatige intelligentie (AI) en computer vision. VGT, de taal die door de Vlaamse dovengemeenschap wordt gebruikt, is een volwaardige taal die, net als andere gesproken talen zoals Nederlands en Frans, erkend wordt door de taalkundige gemeenschap. Het is belangrijk op te merken dat VGT, net als gesproken talen, regionale variaties kent, wat de complexiteit van de vertaling vergroot \autocite{vanmeerbergen2000simultane}.

De structuur van VGT is hiërarchisch, wat betekent dat elk gebaar kan variëren in betekenis afhankelijk van specifieke details, zoals handvorm, beweging en gezichtsuitdrukkingen. \autocite{469340}
Deze variabiliteit maakt het noodzakelijk om geavanceerde algoritmen te ontwikkelen die in staat zijn om deze nuances te herkennen en correct om te zetten naar tekst. 
Recent onderzoek heeft aangetoond dat AI-technologieën, zoals deep learning en convolutionele neurale netwerken (CNN), effectief kunnen worden ingezet voor gebarentaalherkenning.\autocite{10.52756/ijerr.2023.v34spl.004}\autocite{10.17485/ijst/v16i45.2583}
Door gebruik te maken van camera's op smartphones en andere apparaten kan een systeem worden ontwikkeld dat in real-time gebaren herkent en omzet in geschreven tekst.
\section{Literatuurstudie}%
\label{sec:literatuurstudie}

% Voor literatuurverwijzingen zijn er twee belangrijke commando's:
% \autocite{KEY} => (Auteur, jaartal) Gebruik dit als de naam van de auteur
%   geen onderdeel is van de zin.
% \textcite{KEY} => Auteur (jaartal)  Gebruik dit als de auteursnaam wel een
%   functie heeft in de zin (bv. ``Uit onderzoek door Doll & Hill (1954) bleek
%   ...'')

%---------- Methodologie ------------------------------------------------------
\section{Methodologie}%
\label{sec:methodologie}


%---------- Verwachte resultaten ----------------------------------------------
\section{Verwacht resultaat, conclusie}%
\label{sec:verwachte_resultaten}


\printbibliography[heading=bibintoc]

\end{document}