%==============================================================================
% Sjabloon onderzoeksvoorstel bachproef
%==============================================================================
% Gebaseerd op document class `hogent-article'
% zie <https://github.com/HoGentTIN/latex-hogent-article>

% Voor een voorstel in het Engels: voeg de documentclass-optie [english] toe.
% Let op: kan enkel na toestemming van de bachelorproefcoördinator!
\documentclass{hogent-article}

% Invoegen bibliografiebestand
\addbibresource{voorstel.bib}

% Informatie over de opleiding, het vak en soort opdracht
\studyprogramme{Professionele bachelor toegepaste informatica}
\course{Bachelorproef}
\assignmenttype{Onderzoeksvoorstel}
% Voor een voorstel in het Engels, haal de volgende 3 regels uit commentaar
% \studyprogramme{Bachelor of applied information technology}
% \course{Bachelor thesis}
% \assignmenttype{Research proposal}

\academicyear{2024-2025} 

\title{Real-Time Vertaling van Vlaamse Gebarentaal naar Tekst met Computer Vision}

\author{Tom Deganck}
\email{tom.deganck@student.hogent.be}


\supervisor[Co-promotor]{Kim Van Mele}

% Binnen welke specialisatierichting uit 3TI situeert dit onderzoek zich?
% Kies uit deze lijst:
%
% - Mobile \& Enterprise development
% - AI \& Data Engineering
% - Functional \& Business Analysis
% - System \& Network Administrator
% - Mainframe Expert
% - Als het onderzoek niet past binnen een van deze domeinen specifieer je deze
%   zelf
%
\specialisation{AI \& Data Engineering}
\keywords{Kunstmatige Intelligentie, Computer vision,Gebarentaal, Vertalen}

\begin{document}

\begin{abstract}
In de laatste jaren is de behoefte aan technologie die de toegankelijkheid voor doven en slechthorenden vergroot, sterk toegenomen. Een van de belangrijkste uitdagingen waarmee deze gemeenschap geconfronteerd wordt, is de communicatie met de horende wereld. Dit onderzoek richt zich op het verkleinen van deze kloof door de ontwikkeling van een systeem dat Vlaamse Gebarentaal (VGT) in real-time vertaalt naar tekst.

Hiervoor wordt gebruik gemaakt van geavanceerde technieken op het gebied van computer vision en deep learning. Het doel is om een bestaand neuraal netwerk verder te optimaliseren, zodat het gebaren uit videobeelden nog nauwkeuriger herkent en omzet in tekst. Door gebruik te maken van deep learning-modellen zoals convolutionele neurale netwerken (CNN) en transfer learning, zal het vertaalsysteem verbeterd worden om verkeerde classificaties tegen te gaan.
  
De centrale onderzoeksvraag is: "Hoe kunnen reeds bestaand AI-technologieën worden geoptmaliseerd voor de  vertaling van Vlaams Gebarentaal naar tekst?" Deelvragen richten zich op de effectiviteit van bestaande AI-modellen, de uitdagingen bij het ontwikkelen van een vertaalsysteem voor een visueel-ruimtelijke taal, en optimalisatietechnieken voor real-time prestaties.
  
De verwachte resultaten omvatten een systeem dat zowel nauwkeurig als robuust functioneert. De inzichten uit dit onderzoek leveren niet alleen een bijdrage aan de wetenschappelijke kennis van gebarentaalherkenning, maar bieden ook een fundament voor bredere toepassingen in andere gebarentalen wereldwijd. Dit werk illustreert de potentie van AI en assistieve technologie om inclusiviteit te bevorderen en communicatiebarrières tussen gemeenschappen te overbruggen.
\end{abstract}

\tableofcontents

% De hoofdtekst van het voorstel zit in een apart bestand, zodat het makkelijk
% kan opgenomen worden in de bijlagen van de bachelorproef zelf.
%---------- Inleiding ---------------------------------------------------------

% TODO: Is dit voorstel gebaseerd op een paper van Research Methods die je
% vorig jaar hebt ingediend? Heb je daarbij eventueel samengewerkt met een
% andere student?
% Zo ja, haal dan de tekst hieronder uit commentaar en pas aan.

%\paragraph{Opmerking}

% Dit voorstel is gebaseerd op het onderzoeksvoorstel dat werd geschreven in het
% kader van het vak Research Methods dat ik (vorig/dit) academiejaar heb
% uitgewerkt (met medesturent VOORNAAM NAAM als mede-auteur).
% 
\section{Inleiding}%
\label{sec:inleiding}

\subsection{Vlaams Gebarentaal}
\label{sec:VGT}

In de hedendaagse samenleving is er een groeiende behoefte aan technologieën die de communicatie tussen doven en horenden vergemakkelijken. Een van de meest veelbelovende ontwikkelingen op dit gebied is de real-time vertaling van Vlaamse Gebarentaal (VGT) naar tekst, waarbij gebruik wordt gemaakt van kunstmatige intelligentie (AI) en computer vision. VGT, de taal die door de Vlaamse dovengemeenschap wordt gebruikt, is een volwaardige taal die, net als andere gesproken talen zoals Nederlands en Frans, erkend wordt door de taalkundige gemeenschap. Het is belangrijk op te merken dat VGT, net als gesproken talen, regionale variaties kent, wat de complexiteit van de vertaling vergroot \autocite{vanmeerbergen2000simultane}.

De structuur van VGT is hiërarchisch, wat betekent dat elk gebaar kan variëren in betekenis afhankelijk van specifieke details, zoals handvorm, beweging en gezichtsuitdrukkingen. \autocite{469340}
Deze variabiliteit maakt het noodzakelijk om geavanceerde algoritmen te ontwikkelen die in staat zijn om deze nuances te herkennen en correct om te zetten naar tekst. 
Recent onderzoek heeft aangetoond dat AI-technologieën, zoals deep learning en convolutionele neurale netwerken (CNN), effectief kunnen worden ingezet voor gebarentaalherkenning.\autocite{10.52756/ijerr.2023.v34spl.004}\autocite{10.17485/ijst/v16i45.2583}
Door gebruik te maken van camera's op smartphones en andere apparaten kan een systeem worden ontwikkeld dat in real-time gebaren herkent en omzet in geschreven tekst.
\section{Literatuurstudie}%
\label{sec:literatuurstudie}

% Voor literatuurverwijzingen zijn er twee belangrijke commando's:
% \autocite{KEY} => (Auteur, jaartal) Gebruik dit als de naam van de auteur
%   geen onderdeel is van de zin.
% \textcite{KEY} => Auteur (jaartal)  Gebruik dit als de auteursnaam wel een
%   functie heeft in de zin (bv. ``Uit onderzoek door Doll & Hill (1954) bleek
%   ...'')

%---------- Methodologie ------------------------------------------------------
\section{Methodologie}%
\label{sec:methodologie}


%---------- Verwachte resultaten ----------------------------------------------
\section{Verwacht resultaat, conclusie}%
\label{sec:verwachte_resultaten}


\printbibliography[heading=bibintoc]

\end{document}