%---------- Inleiding ---------------------------------------------------------

% TODO: Is dit voorstel gebaseerd op een paper van Research Methods die je
% vorig jaar hebt ingediend? Heb je daarbij eventueel samengewerkt met een
% andere student?
% Zo ja, haal dan de tekst hieronder uit commentaar en pas aan.

%\paragraph{Opmerking}

% Dit voorstel is gebaseerd op het onderzoeksvoorstel dat werd geschreven in het
% kader van het vak Research Methods dat ik (vorig/dit) academiejaar heb
% uitgewerkt (met medesturent VOORNAAM NAAM als mede-auteur).
% 
\section{Inleiding}%
\label{sec:inleiding}

\subsection{Vlaams Gebarentaal}
\label{sec:VGT}

Het Vlaams Gebarentaal (VGT) is de taal die door de Vlaamse Dovengemeenschap wordt gebruikt. Net als Nederlands, Frans of Engels is VGT een volwaardige taal en wordt het erkend door de taalkundige gemeenschap.\autocite{vanmeerbergen2000simultane} Toch kent VGT, net zoals gesproken talen, regionale variaties.\autocite{10.1093/ijl/ecy008}

Elk gebaar in VGT is opgebouwd uit een hiërarchische boomstructuur, waardoor meerdere gebaren een andere betekenis kunnen hebben afhankelijk van bepaalde details.\autocite{469340}

\section{Literatuurstudie}%
\label{sec:literatuurstudie}

% Voor literatuurverwijzingen zijn er twee belangrijke commando's:
% \autocite{KEY} => (Auteur, jaartal) Gebruik dit als de naam van de auteur
%   geen onderdeel is van de zin.
% \textcite{KEY} => Auteur (jaartal)  Gebruik dit als de auteursnaam wel een
%   functie heeft in de zin (bv. ``Uit onderzoek door Doll & Hill (1954) bleek
%   ...'')

%---------- Methodologie ------------------------------------------------------
\section{Methodologie}%
\label{sec:methodologie}


%---------- Verwachte resultaten ----------------------------------------------
\section{Verwacht resultaat, conclusie}%
\label{sec:verwachte_resultaten}
