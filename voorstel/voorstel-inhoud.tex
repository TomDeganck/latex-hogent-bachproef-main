%---------- Inleiding ---------------------------------------------------------

% TODO: Is dit voorstel gebaseerd op een paper van Research Methods die je
% vorig jaar hebt ingediend? Heb je daarbij eventueel samengewerkt met een
% andere student?
% Zo ja, haal dan de tekst hieronder uit commentaar en pas aan.

%\paragraph{Opmerking}

% Dit voorstel is gebaseerd op het onderzoeksvoorstel dat werd geschreven in het
% kader van het vak Research Methods dat ik (vorig/dit) academiejaar heb
% uitgewerkt (met medesturent VOORNAAM NAAM als mede-auteur).
% 
\section{Inleiding}%
\label{sec:inleiding}

In de hedendaagse samenleving is er een groeiende behoefte aan technologieën die de communicatie tussen doven en horenden vergemakkelijken. Een van de meest veelbelovende ontwikkelingen op dit gebied is de real-time vertaling van Vlaamse Gebarentaal (VGT) naar tekst, waarbij gebruik wordt gemaakt van kunstmatige intelligentie (AI) en computer vision. VGT, de taal die door de Vlaamse dovengemeenschap wordt gebruikt, is een volwaardige taal die, net als andere gesproken talen zoals Nederlands en Frans, erkend wordt door de taalkundige gemeenschap. Het is belangrijk op te merken dat VGT, net als gesproken talen, regionale variaties kent, wat de complexiteit van de vertaling vergroot \autocite{vanmeerbergen2000simultane}.

De structuur van VGT is hiërarchisch, wat betekent dat elk gebaar kan variëren in betekenis afhankelijk van specifieke details, zoals handvorm, beweging en gezichtsuitdrukkingen. \autocite{469340}
Deze variabiliteit maakt het noodzakelijk om geavanceerde algoritmen te ontwikkelen die in staat zijn om deze nuances te herkennen en correct om te zetten naar tekst. 
Recent onderzoek heeft aangetoond dat AI-technologieën, zoals deep learning en convolutionele neurale netwerken (CNN), effectief kunnen worden ingezet voor gebarentaalherkenning.\autocite{10.52756/ijerr.2023.v34spl.004}\autocite{10.17485/ijst/v16i45.2583}
Door gebruik te maken van camera's op smartphones en andere apparaten kan een systeem worden ontwikkeld dat in real-time gebaren herkent en omzet in geschreven tekst.
Deze technologische vooruitgang belooft een aanzienlijke impact te hebben op de toegankelijkheid en inclusie van de dove gemeenschap, waardoor zij beter kunnen communiceren met de horende wereld. Dit document onderzoekt de huidige stand van zaken op het gebied van AI-gestuurde VGT-herkenning en de potentiële toepassingen en uitdagingen die gepaard gaan met de implementatie van dergelijke systemen.

\section{Literatuurstudie}%
\label{sec:literatuurstudie}

% Voor literatuurverwijzingen zijn er twee belangrijke commando's:
% \autocite{KEY} => (Auteur, jaartal) Gebruik dit als de naam van de auteur
%   geen onderdeel is van de zin.
% \textcite{KEY} => Auteur (jaartal)  Gebruik dit als de auteursnaam wel een
%   functie heeft in de zin (bv. ``Uit onderzoek door Doll & Hill (1954) bleek
%   ...'')

%---------- Methodologie ------------------------------------------------------
\section{Methodologie}%
\label{sec:methodologie}
\subsection{Fase 1: Data-acquisitie}
In de eerste en meest cruciale fase van het onderzoek zal de benodigde data worden verzameld.
Voor het verzamelen van de data zal gebruik worden gemaakt van verschillende bronnen.
Deze bronnen zullen bestaan uit bestaande datasets, zoals de VGT-dataset van de Universiteit van Gent.
Daarnaast zullen er ook nieuwe datasets worden gecreëerd door middel van het opnemen van VGT-gebaren met behulp van een camera.
Deze datasets zullen worden gebruikt om het model te trainen en te valideren.
\subsection{Fase 2: Preprocessing}
\subsection{Fase 3: Modeltraining}
\subsection{Fase 4: Real-time implementatie}
\subsection{Fase 5: Evaluatie \& Validatie}
%---------- Verwachte resultaten ----------------------------------------------
\section{Verwacht resultaat, conclusie}%
\label{sec:verwachte_resultaten}
