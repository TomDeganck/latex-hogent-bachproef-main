%%=============================================================================
%% Inleiding
%%=============================================================================

\chapter{\IfLanguageName{dutch}{Inleiding}{Introduction}}%
\label{ch:inleiding}

De behoefte aan effectieve communicatie tussen dove en horende mensen is in de afgelopen jaren steeds groter geworden. 
Gebarentaal is voor veel mensen in de doven en slechthorenden gemeenschap de primaire communicatievorm. 
Het vertalen van gebarentaal naar tekst kan deze communicatiekloof overbruggen, en nieuwe technologieën, zoals deep learning, bieden kansen om dit proces te automatiseren en real-time te Maken. 
Dit onderzoek richt zich op de toepassing van deep learning-modellen voor de real-time vertaling van een Gebarentaal naar tekst.

Het onderwerp van dit onderzoek is relevant gezien de noodzaak om technologieën te ontwikkelen die sociale inclusie bevorderen. 
Doordat de samenleving steeds digitaler wordt, is het van groot belang om oplossingen te vinden die de communicatie tussen doven en horenden vergemakkelijken, zeker in contexten zoals scholen, werkplekken en openbare instellingen. 
Dit onderzoek beoogt een diepgaand inzicht te verkrijgen in de mogelijkheden en beperkingen van automatische gebarentaalherkenningstechnologieën.

De methodologie van dit onderzoek is gebaseerd op het gebruik van een pre-getraind deep learning model voor gebarentaalherkenning, met de focus op de implementatie van een efficiënte oplossing die geschikt is voor mobiele apparaten. 
De keuze van het platform en de inzet van het model in een mobiele applicatie vormen kernaspecten van dit onderzoek. 
Daarbij zal er bijzondere aandacht zijn voor de afweging tussen on-device en cloud-based deployment en de impact daarvan op latentie, prestaties en toegankelijkheid.

\section{\IfLanguageName{dutch}{Probleemstelling}{Problem Statement}}%
\label{sec:probleemstelling}

In veel situaties kunnen doven en slechthorenden niet rekenen op geschikte hulpmiddelen voor communicatie in Vlaamse Gebarentaal.
De ontwikkeling van een betrouwbare en efficiënte technologie voor het vertalen van gebarentaal naar tekst kan een belangrijke bijdrage leveren aan het verbeteren van de communicatie tussen verschillende groepen binnen de samenleving. 
Het probleem dat dit onderzoek adresseert, is het creëren van een real-time vertaalmodel voor Vlaamse Gebarentaal naar tekst dat goed presteert op mobiele apparaten, waarbij rekening wordt gehouden met de technische en praktische beperkingen van dergelijke apparaten.

De doelgroep van dit onderzoek bestaat voornamelijk uit gebruikers van gebarentaal, zoals doven en slechthorenden, en organisaties die hen ondersteunen. 
Dit kan zowel gaan om onderwijsinstellingen als openbare voorzieningen die communicatie met doven vergemakkelijken. 
Het doel van het onderzoek is om een proof-of-concept te ontwikkelen dat als basis kan dienen voor verdere toepassingen in de praktijk.

\section{\IfLanguageName{dutch}{Onderzoeksvraag}{Research question}}%
\label{sec:onderzoeksvraag}

De centrale onderzoeksvraag in dit onderzoek is:
\begin{quote}
    Hoe kan een deep learning-model voor Vlaamse Gebarentaal effectief worden ingezet voor real-time vertaling naar tekst op mobiele apparaten, rekening houdend met de beperkingen van latentie, prestaties en toegankelijkheid?
\end{quote}

Daarnaast worden de volgende deelvragen geformuleerd:
\begin{itemize}
  \item Wat zijn de technische eisen voor het ontwikkelen van een deep learning-model dat gebarentaal naar tekst vertaalt?
  \item Welke deployment-opties zijn het meest geschikt voor real-time verwerking op mobiele apparaten?
  \item Wat zijn de prestaties van het model in termen van latentie en nauwkeurigheid, zowel bij on-device als cloud-gebaseerde implementaties?
\end{itemize}

\section{\IfLanguageName{dutch}{Onderzoeksdoelstelling}{Research objective}}
\label{sec:onderzoeksdoelstelling}

Het doel van dit onderzoek is het ontwikkelen van een proof-of-concept dat in staat is om Vlaamse Gebarentaal real-time te vertalen naar tekst op een mobiel apparaat. 
Dit moet gebeuren door het gebruik van een deep learning-model dat vooraf is getraind en geoptimaliseerd voor gebruik op een mobiele omgeving. 
De succescriteria voor dit onderzoek zijn:
\begin{itemize}
  \item Het model kan gebarentaal vertalen naar tekst.
  \item Het systeem werkt met lage latentie en in real-time.
  \item Het model is efficiënt qua gebruik van rekenkracht en opslag op mobiele apparaten.
\end{itemize}

\section{\IfLanguageName{dutch}{Opzet van deze bachelorproef}{Structure of this bachelor thesis}}
\label{sec:opzet-bachelorproef}

De rest van deze bachelorproef is als volgt opgebouwd:

In Hoofdstuk~\ref{ch:stand-van-zaken} wordt de stand van zaken binnen het onderzoeksdomein besproken, waarbij bestaande technologieën en benaderingen voor gebarentaalherkenning worden behandeld.

In Hoofdstuk~\ref{ch:methodologie} wordt de gebruikte methodologie beschreven, met specifieke aandacht voor de keuze van het deep learning-model, de technische implementatie en de deployment-strategieën.

In Hoofdstuk~\ref{ch:conclusie} worden de conclusies getrokken en wordt een antwoord geformuleerd op de onderzoeksvraag. Tevens wordt er een blik geworpen op mogelijke verbeteringen en toekomstig onderzoek op dit gebied.
