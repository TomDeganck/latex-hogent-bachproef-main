%%=============================================================================
%% Voorwoord
%%=============================================================================

\chapter*{\IfLanguageName{dutch}{Woord vooraf}{Preface}}%
\label{ch:voorwoord}

%% TODO:
%% Het voorwoord is het enige deel van de bachelorproef waar je vanuit je
%% eigen standpunt (``ik-vorm'') mag schrijven. Je kan hier bv. motiveren
%% waarom jij het onderwerp wil bespreken.
%% Vergeet ook niet te bedanken wie je geholpen/gesteund/... heeft

De afronding van deze bachelorproef markeert het einde van een intensieve en leerrijke periode vol onderzoek en ontwikkeling. Gedurende dit traject heb ik mij gericht op het vertalen van gebarentaal met behulp van computer vision en de uitdagingen die komen kijken bij de implementatie van een vertaalmodel op een mobiel apparaat.
\\
\\
Dit project vergde niet alleen een diepgaande duik in de materie, maar ook het oplossen van complexe vraagstukken om de brug te slaan tussen theoretische modellen en praktische implementatie op een mobiel platform. Het vroeg om doorzettingsvermogen en het verkennen van nieuwe paden, wat heeft geleid tot waardevolle inzichten en persoonlijke groei op technisch, analytisch en probleemoplossend gebied.
\\
\\
Graag wil ik de personen bedanken die essentieel waren voor de realisatie van deze proef. Allereerst gaat mijn oprechte dank uit naar mijn promotor, Stijn Lievens, en mijn co-promotor, Kim Van Mele. Hun deskundige begeleiding, geduld, advies en kritische feedback hebben de koers van dit onderzoek mede bepaald en mij geholpen struikelblokken te overwinnen. Hun steun werd enorm gewaardeerd.
\\
\\
Een speciaal woord van dank ook aan mijn partner, voor de onvoorwaardelijke steun en het begrip gedurende de drukke periodes die dit project met zich meebracht.
\\
\\
Mijn motivatie voor dit onderwerp lag mede in het vergemakkelijken van communicatie tussen de doven/slechthorenden gemeenschap en horende sprekers.